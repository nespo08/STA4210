% Options for packages loaded elsewhere
\PassOptionsToPackage{unicode}{hyperref}
\PassOptionsToPackage{hyphens}{url}
%
\documentclass[
]{article}
\usepackage{amsmath,amssymb}
\usepackage{lmodern}
\usepackage{ifxetex,ifluatex}
\ifnum 0\ifxetex 1\fi\ifluatex 1\fi=0 % if pdftex
  \usepackage[T1]{fontenc}
  \usepackage[utf8]{inputenc}
  \usepackage{textcomp} % provide euro and other symbols
\else % if luatex or xetex
  \usepackage{unicode-math}
  \defaultfontfeatures{Scale=MatchLowercase}
  \defaultfontfeatures[\rmfamily]{Ligatures=TeX,Scale=1}
\fi
% Use upquote if available, for straight quotes in verbatim environments
\IfFileExists{upquote.sty}{\usepackage{upquote}}{}
\IfFileExists{microtype.sty}{% use microtype if available
  \usepackage[]{microtype}
  \UseMicrotypeSet[protrusion]{basicmath} % disable protrusion for tt fonts
}{}
\makeatletter
\@ifundefined{KOMAClassName}{% if non-KOMA class
  \IfFileExists{parskip.sty}{%
    \usepackage{parskip}
  }{% else
    \setlength{\parindent}{0pt}
    \setlength{\parskip}{6pt plus 2pt minus 1pt}}
}{% if KOMA class
  \KOMAoptions{parskip=half}}
\makeatother
\usepackage{xcolor}
\IfFileExists{xurl.sty}{\usepackage{xurl}}{} % add URL line breaks if available
\IfFileExists{bookmark.sty}{\usepackage{bookmark}}{\usepackage{hyperref}}
\hypersetup{
  pdftitle={Homework 2},
  pdfauthor={Nicholas Esposito},
  hidelinks,
  pdfcreator={LaTeX via pandoc}}
\urlstyle{same} % disable monospaced font for URLs
\usepackage[margin=1in]{geometry}
\usepackage{color}
\usepackage{fancyvrb}
\newcommand{\VerbBar}{|}
\newcommand{\VERB}{\Verb[commandchars=\\\{\}]}
\DefineVerbatimEnvironment{Highlighting}{Verbatim}{commandchars=\\\{\}}
% Add ',fontsize=\small' for more characters per line
\usepackage{framed}
\definecolor{shadecolor}{RGB}{248,248,248}
\newenvironment{Shaded}{\begin{snugshade}}{\end{snugshade}}
\newcommand{\AlertTok}[1]{\textcolor[rgb]{0.94,0.16,0.16}{#1}}
\newcommand{\AnnotationTok}[1]{\textcolor[rgb]{0.56,0.35,0.01}{\textbf{\textit{#1}}}}
\newcommand{\AttributeTok}[1]{\textcolor[rgb]{0.77,0.63,0.00}{#1}}
\newcommand{\BaseNTok}[1]{\textcolor[rgb]{0.00,0.00,0.81}{#1}}
\newcommand{\BuiltInTok}[1]{#1}
\newcommand{\CharTok}[1]{\textcolor[rgb]{0.31,0.60,0.02}{#1}}
\newcommand{\CommentTok}[1]{\textcolor[rgb]{0.56,0.35,0.01}{\textit{#1}}}
\newcommand{\CommentVarTok}[1]{\textcolor[rgb]{0.56,0.35,0.01}{\textbf{\textit{#1}}}}
\newcommand{\ConstantTok}[1]{\textcolor[rgb]{0.00,0.00,0.00}{#1}}
\newcommand{\ControlFlowTok}[1]{\textcolor[rgb]{0.13,0.29,0.53}{\textbf{#1}}}
\newcommand{\DataTypeTok}[1]{\textcolor[rgb]{0.13,0.29,0.53}{#1}}
\newcommand{\DecValTok}[1]{\textcolor[rgb]{0.00,0.00,0.81}{#1}}
\newcommand{\DocumentationTok}[1]{\textcolor[rgb]{0.56,0.35,0.01}{\textbf{\textit{#1}}}}
\newcommand{\ErrorTok}[1]{\textcolor[rgb]{0.64,0.00,0.00}{\textbf{#1}}}
\newcommand{\ExtensionTok}[1]{#1}
\newcommand{\FloatTok}[1]{\textcolor[rgb]{0.00,0.00,0.81}{#1}}
\newcommand{\FunctionTok}[1]{\textcolor[rgb]{0.00,0.00,0.00}{#1}}
\newcommand{\ImportTok}[1]{#1}
\newcommand{\InformationTok}[1]{\textcolor[rgb]{0.56,0.35,0.01}{\textbf{\textit{#1}}}}
\newcommand{\KeywordTok}[1]{\textcolor[rgb]{0.13,0.29,0.53}{\textbf{#1}}}
\newcommand{\NormalTok}[1]{#1}
\newcommand{\OperatorTok}[1]{\textcolor[rgb]{0.81,0.36,0.00}{\textbf{#1}}}
\newcommand{\OtherTok}[1]{\textcolor[rgb]{0.56,0.35,0.01}{#1}}
\newcommand{\PreprocessorTok}[1]{\textcolor[rgb]{0.56,0.35,0.01}{\textit{#1}}}
\newcommand{\RegionMarkerTok}[1]{#1}
\newcommand{\SpecialCharTok}[1]{\textcolor[rgb]{0.00,0.00,0.00}{#1}}
\newcommand{\SpecialStringTok}[1]{\textcolor[rgb]{0.31,0.60,0.02}{#1}}
\newcommand{\StringTok}[1]{\textcolor[rgb]{0.31,0.60,0.02}{#1}}
\newcommand{\VariableTok}[1]{\textcolor[rgb]{0.00,0.00,0.00}{#1}}
\newcommand{\VerbatimStringTok}[1]{\textcolor[rgb]{0.31,0.60,0.02}{#1}}
\newcommand{\WarningTok}[1]{\textcolor[rgb]{0.56,0.35,0.01}{\textbf{\textit{#1}}}}
\usepackage{graphicx}
\makeatletter
\def\maxwidth{\ifdim\Gin@nat@width>\linewidth\linewidth\else\Gin@nat@width\fi}
\def\maxheight{\ifdim\Gin@nat@height>\textheight\textheight\else\Gin@nat@height\fi}
\makeatother
% Scale images if necessary, so that they will not overflow the page
% margins by default, and it is still possible to overwrite the defaults
% using explicit options in \includegraphics[width, height, ...]{}
\setkeys{Gin}{width=\maxwidth,height=\maxheight,keepaspectratio}
% Set default figure placement to htbp
\makeatletter
\def\fps@figure{htbp}
\makeatother
\setlength{\emergencystretch}{3em} % prevent overfull lines
\providecommand{\tightlist}{%
  \setlength{\itemsep}{0pt}\setlength{\parskip}{0pt}}
\setcounter{secnumdepth}{-\maxdimen} % remove section numbering
\ifluatex
  \usepackage{selnolig}  % disable illegal ligatures
\fi

\title{Homework 2}
\author{Nicholas Esposito}
\date{10/9/2022}

\begin{document}
\maketitle

\begin{verbatim}
## Loading required package: zoo
\end{verbatim}

\begin{verbatim}
## 
## Attaching package: 'zoo'
\end{verbatim}

\begin{verbatim}
## The following objects are masked from 'package:base':
## 
##     as.Date, as.Date.numeric
\end{verbatim}

\hypertarget{load-the-data}{%
\subsubsection{Load the data}\label{load-the-data}}

Y - Deflection of galvonometer, X - Area of the wires on the coupling

\hypertarget{a-the-fitted-equation-and-residuals}{%
\subsubsection{a) The fitted equation and
residuals}\label{a-the-fitted-equation-and-residuals}}

The fitted equation for the simple linear model relating the deflection
of galvonometer (Y) to the area of the wires on the coupling (X) is:

Y = 184.436 - 0.695X

\begin{verbatim}
## 
## Call:
## lm(formula = Y ~ X)
## 
## Residuals:
##     Min      1Q  Median      3Q     Max 
## -10.042  -6.459  -0.068   2.901  19.386 
## 
## Coefficients:
##              Estimate Std. Error t value Pr(>|t|)    
## (Intercept) 184.43569    2.91526   63.27  < 2e-16 ***
## X            -0.69537    0.03383  -20.56 6.39e-15 ***
## ---
## Signif. codes:  0 '***' 0.001 '**' 0.01 '*' 0.05 '.' 0.1 ' ' 1
## 
## Residual standard error: 7.337 on 20 degrees of freedom
## Multiple R-squared:  0.9548, Adjusted R-squared:  0.9526 
## F-statistic: 422.6 on 1 and 20 DF,  p-value: 6.386e-15
\end{verbatim}

\begin{verbatim}
##                      b1       b0
## Fitted equation -0.6954 184.4357
\end{verbatim}

The residuals are:

\begin{verbatim}
##            1            2            3            4            5            6 
##   6.25989890   2.75989890   4.75989890   4.48502508  -7.01497492   0.98502508 
##            7            8            9           10           11           12 
##  -6.59448305  -0.09448305  -7.54155105 -10.04155105  -0.04155105   1.41869488 
##           13           14           15           16           17           18 
##   1.41869488  -0.58130512  -7.59788767  -1.59788767  -9.09788767  -6.05154464 
##           19           20           21           22 
##   2.94845536  -1.05154464  12.88552978  19.38552978
\end{verbatim}

\hypertarget{b-plot-of-y-vs-x}{%
\subsubsection{b) Plot of Y vs X}\label{b-plot-of-y-vs-x}}

\begin{Shaded}
\begin{Highlighting}[]
\FunctionTok{plot}\NormalTok{(Y }\SpecialCharTok{\textasciitilde{}}\NormalTok{ X, }\AttributeTok{main =} \StringTok{"Deflection of Galvonomete (Y) vs Area of the Wires on the Coupling (X)"}\NormalTok{)}
\FunctionTok{abline}\NormalTok{(lm\_exp\_data)}
\end{Highlighting}
\end{Shaded}

\includegraphics{Homework-2_files/figure-latex/unnamed-chunk-5-1.pdf}

\begin{enumerate}
\def\labelenumi{\alph{enumi})}
\setcounter{enumi}{2}
\tightlist
\item
  Residual Plot of e vs X
\end{enumerate}

\begin{Shaded}
\begin{Highlighting}[]
\FunctionTok{plot}\NormalTok{(X, residuals, }\AttributeTok{main =} \StringTok{"Residuals vs Predictors (X)"}\NormalTok{)}
\FunctionTok{abline}\NormalTok{(lm\_exp\_data)}
\end{Highlighting}
\end{Shaded}

\includegraphics{Homework-2_files/figure-latex/unnamed-chunk-6-1.pdf}

\begin{enumerate}
\def\labelenumi{\alph{enumi})}
\setcounter{enumi}{3}
\tightlist
\item
  A test of whether deflections (Y) are associated with area of the
  wires (X)
\end{enumerate}

Use Pearson's correlation test to test H0: p = 0 versus Ha: p != 0
(alpha=0.05).

Test-Statistic (t\emph{): -20.557 Critical Value, or t(0.975, 20): 2.086
Rejection Region: \textbar t}\textbar{} \textgreater= t(0.975, 20)

Since \textbar-20.557\textbar{} \textgreater= 2.086, the test statistic
t* is statistically significant at alpha = 0.05. Thus, we reject the
null hypothesis H0: p = 0 and conclude that there is a linear
association between deflections (Y) and area of the wires (X).

\begin{Shaded}
\begin{Highlighting}[]
\CommentTok{\# Calculation by hand}
\NormalTok{SSyy }\OtherTok{\textless{}{-}} \FunctionTok{sum}\NormalTok{((Y}\SpecialCharTok{{-}}\NormalTok{ybar)}\SpecialCharTok{\^{}}\DecValTok{2}\NormalTok{)}
\NormalTok{R }\OtherTok{\textless{}{-}}\NormalTok{ SSxy }\SpecialCharTok{/} \FunctionTok{sqrt}\NormalTok{(SSxx}\SpecialCharTok{*}\NormalTok{SSyy)}
\NormalTok{t.Cor }\OtherTok{\textless{}{-}}\NormalTok{ R}\SpecialCharTok{*}\FunctionTok{sqrt}\NormalTok{(n}\DecValTok{{-}2}\NormalTok{) }\SpecialCharTok{/} \FunctionTok{sqrt}\NormalTok{(}\DecValTok{1}\SpecialCharTok{{-}}\NormalTok{R}\SpecialCharTok{\^{}}\DecValTok{2}\NormalTok{)}

\NormalTok{cor.out }\OtherTok{\textless{}{-}} \FunctionTok{cbind}\NormalTok{(R,t.Cor)}
\FunctionTok{colnames}\NormalTok{(cor.out) }\OtherTok{\textless{}{-}} \FunctionTok{c}\NormalTok{(}\StringTok{"R"}\NormalTok{, }\StringTok{"t*"}\NormalTok{)}
\FunctionTok{rownames}\NormalTok{(cor.out) }\OtherTok{\textless{}{-}} \FunctionTok{c}\NormalTok{(}\StringTok{"Correlation Test"}\NormalTok{)}
\FunctionTok{round}\NormalTok{(cor.out, }\DecValTok{4}\NormalTok{)}
\end{Highlighting}
\end{Shaded}

\begin{verbatim}
##                        R       t*
## Correlation Test -0.9771 -20.5573
\end{verbatim}

\begin{enumerate}
\def\labelenumi{\alph{enumi})}
\setcounter{enumi}{4}
\tightlist
\item
  Normal Probability Plot of the residuals
\end{enumerate}

\begin{Shaded}
\begin{Highlighting}[]
\FunctionTok{qqnorm}\NormalTok{(residuals)}
\FunctionTok{qqline}\NormalTok{(residuals)}
\end{Highlighting}
\end{Shaded}

\includegraphics{Homework-2_files/figure-latex/unnamed-chunk-8-1.pdf}

\begin{enumerate}
\def\labelenumi{\alph{enumi})}
\setcounter{enumi}{5}
\tightlist
\item
  Shapiro-Wilk test for normality
\end{enumerate}

H0: The errors follow a normal distribution.

Ha: The errors do not follow a normal distribution.

The Shapiro-Wilk test produces a p-value of 0.09. Since 0.09
\textgreater{} 0.05, we fail to reject H0, and thus conclude that the
residuals follow a normal distribution at the significance level 0.05.

\begin{Shaded}
\begin{Highlighting}[]
\FunctionTok{shapiro.test}\NormalTok{(residuals)}
\end{Highlighting}
\end{Shaded}

\begin{verbatim}
## 
##  Shapiro-Wilk normality test
## 
## data:  residuals
## W = 0.92354, p-value = 0.09004
\end{verbatim}

\begin{enumerate}
\def\labelenumi{\alph{enumi})}
\setcounter{enumi}{6}
\tightlist
\item
  Brown-Forsyth test for constant variance
\end{enumerate}

H0: There is equal variance among the errors.

Ha: There is unequal variance among the errors (Increasing or Decreasing
in X).

We have test statistic t* = -0.5534, and will compare it against
t(0.975, 20) = 2.086.

Since \textbar t*\textbar{} = 0.5534 \textless{} 2.086, we fail to
reject H0, and thus conclude that there is equal variance among the
errors at the significance level 0.05.

\begin{Shaded}
\begin{Highlighting}[]
\NormalTok{group.BF }\OtherTok{\textless{}{-}} \FunctionTok{ifelse}\NormalTok{(exp\_data}\SpecialCharTok{$}\NormalTok{coupling }\SpecialCharTok{\textless{}=} \DecValTok{4}\NormalTok{, }\DecValTok{1}\NormalTok{, }\DecValTok{2}\NormalTok{) }\CommentTok{\# Breaks the data by coupling value}

\NormalTok{fit1 }\OtherTok{\textless{}{-}} \FunctionTok{lm}\NormalTok{(exp\_data}\SpecialCharTok{$}\NormalTok{defGalv }\SpecialCharTok{\textasciitilde{}}\NormalTok{ exp\_data}\SpecialCharTok{$}\NormalTok{areaWire)}
\NormalTok{res1 }\OtherTok{\textless{}{-}} \FunctionTok{resid}\NormalTok{(fit1)}

\NormalTok{median\_e1 }\OtherTok{\textless{}{-}} \FunctionTok{median}\NormalTok{(residuals[group.BF }\SpecialCharTok{==} \DecValTok{1}\NormalTok{]) }\CommentTok{\# Median residuals}
\NormalTok{median\_e2 }\OtherTok{\textless{}{-}} \FunctionTok{median}\NormalTok{(residuals[group.BF }\SpecialCharTok{==} \DecValTok{2}\NormalTok{])}

\NormalTok{median\_e }\OtherTok{\textless{}{-}} \FunctionTok{rep}\NormalTok{(}\FunctionTok{c}\NormalTok{(median\_e1, median\_e2), }\AttributeTok{each =} \DecValTok{11}\NormalTok{)}

\NormalTok{d.BF }\OtherTok{\textless{}{-}} \FunctionTok{abs}\NormalTok{(res1 }\SpecialCharTok{{-}}\NormalTok{ median\_e)}
\FunctionTok{cbind}\NormalTok{(group.BF, d.BF)}
\end{Highlighting}
\end{Shaded}

\begin{verbatim}
##    group.BF        d.BF
## 1         1  6.30144996
## 2         1  2.80144996
## 3         1  4.80144996
## 4         1  4.52657613
## 5         1  6.97342387
## 6         1  1.02657613
## 7         1  6.55293199
## 8         1  0.05293199
## 9         1  7.50000000
## 10        1 10.00000000
## 11        1  0.00000000
## 12        2  2.00000000
## 13        2  2.00000000
## 14        2  0.00000000
## 15        2  7.01658255
## 16        2  1.01658255
## 17        2  8.51658255
## 18        2  5.47023952
## 19        2  3.52976048
## 20        2  0.47023952
## 21        2 13.46683490
## 22        2 19.96683490
\end{verbatim}

\begin{Shaded}
\begin{Highlighting}[]
\CommentTok{\# Brute force calculation }
\NormalTok{dbar1 }\OtherTok{\textless{}{-}} \FunctionTok{mean}\NormalTok{(d.BF[group.BF }\SpecialCharTok{==} \DecValTok{1}\NormalTok{])}
\NormalTok{dbar2 }\OtherTok{\textless{}{-}} \FunctionTok{mean}\NormalTok{(d.BF[group.BF }\SpecialCharTok{==} \DecValTok{2}\NormalTok{])}

\NormalTok{var1 }\OtherTok{\textless{}{-}} \FunctionTok{var}\NormalTok{(d.BF[group.BF }\SpecialCharTok{==} \DecValTok{1}\NormalTok{])}
\NormalTok{var2 }\OtherTok{\textless{}{-}} \FunctionTok{var}\NormalTok{(d.BF[group.BF }\SpecialCharTok{==} \DecValTok{2}\NormalTok{])}

\NormalTok{n1 }\OtherTok{\textless{}{-}} \FunctionTok{length}\NormalTok{(d.BF[group.BF }\SpecialCharTok{==} \DecValTok{1}\NormalTok{])}
\NormalTok{n2 }\OtherTok{\textless{}{-}} \FunctionTok{length}\NormalTok{(d.BF[group.BF }\SpecialCharTok{==} \DecValTok{2}\NormalTok{])}

\NormalTok{var.p }\OtherTok{\textless{}{-}}\NormalTok{ ((n1}\DecValTok{{-}1}\NormalTok{)}\SpecialCharTok{*}\NormalTok{var1 }\SpecialCharTok{+}\NormalTok{ (n2}\DecValTok{{-}1}\NormalTok{)}\SpecialCharTok{*}\NormalTok{var2) }\SpecialCharTok{/}\NormalTok{ (n1}\SpecialCharTok{+}\NormalTok{n2}\DecValTok{{-}2}\NormalTok{) }\CommentTok{\# Pooled variance}

\NormalTok{t.BF }\OtherTok{\textless{}{-}}\NormalTok{ (dbar1 }\SpecialCharTok{{-}}\NormalTok{ dbar2) }\SpecialCharTok{/} \FunctionTok{sqrt}\NormalTok{(var.p}\SpecialCharTok{*}\NormalTok{(}\DecValTok{1}\SpecialCharTok{/}\NormalTok{n1 }\SpecialCharTok{+} \DecValTok{1}\SpecialCharTok{/}\NormalTok{n2)) }\CommentTok{\# T{-}statistic}

\NormalTok{p.t.BF }\OtherTok{\textless{}{-}} \DecValTok{2}\SpecialCharTok{*}\NormalTok{(}\DecValTok{1}\SpecialCharTok{{-}}\FunctionTok{pt}\NormalTok{(}\FunctionTok{abs}\NormalTok{(t.BF), n1}\SpecialCharTok{+}\NormalTok{n2}\DecValTok{{-}2}\NormalTok{)) }\CommentTok{\# P{-}value}

\NormalTok{BF.out }\OtherTok{\textless{}{-}} \FunctionTok{cbind}\NormalTok{(dbar1 }\SpecialCharTok{{-}}\NormalTok{ dbar2, t.BF, p.t.BF)}
\FunctionTok{colnames}\NormalTok{(BF.out) }\OtherTok{\textless{}{-}} \FunctionTok{c}\NormalTok{(}\StringTok{"Mean Diff"}\NormalTok{, }\StringTok{"t*"}\NormalTok{, }\StringTok{"2P(\textgreater{}|t*|))"}\NormalTok{)}
\FunctionTok{rownames}\NormalTok{(BF.out) }\OtherTok{\textless{}{-}} \FunctionTok{c}\NormalTok{(}\StringTok{"BF Test"}\NormalTok{)}
\FunctionTok{round}\NormalTok{(BF.out, }\DecValTok{4}\NormalTok{)}
\end{Highlighting}
\end{Shaded}

\begin{verbatim}
##         Mean Diff      t* 2P(>|t*|))
## BF Test   -1.1743 -0.5534     0.5861
\end{verbatim}

\begin{enumerate}
\def\labelenumi{\alph{enumi})}
\setcounter{enumi}{7}
\tightlist
\item
  Breusch-Pagan Test for constant variance
\end{enumerate}

H0: There is equal variance among the errors.

Ha: There is unequal variance among the errors.

We have test statistic X\^{}2* = 2.5916, and will compare it against
X\^{}2(0.95, 1) = 3.8415.

Since X\^{}2* = 2.5916 \textless{} 3.8415, we fail to reject H0, and
thus conclude that there is equal variance among the errors at the
significance level 0.05.

\begin{Shaded}
\begin{Highlighting}[]
\CommentTok{\# 1. Find ei\^{}2}
\NormalTok{residuals.sq }\OtherTok{\textless{}{-}}\NormalTok{ residuals}\SpecialCharTok{\^{}}\DecValTok{2}

\CommentTok{\# 2. Fit Regression of ei\^{}2 on X}
\NormalTok{lm.BP }\OtherTok{\textless{}{-}} \FunctionTok{lm}\NormalTok{(residuals.sq }\SpecialCharTok{\textasciitilde{}}\NormalTok{ X)}
\FunctionTok{summary}\NormalTok{(lm.BP)}
\end{Highlighting}
\end{Shaded}

\begin{verbatim}
## 
## Call:
## lm(formula = residuals.sq ~ X)
## 
## Residuals:
##    Min     1Q Median     3Q    Max 
## -76.88 -57.45  -5.59  19.19 292.95 
## 
## Coefficients:
##             Estimate Std. Error t value Pr(>|t|)   
## (Intercept)  93.1956    32.0989   2.903  0.00879 **
## X            -0.6085     0.3724  -1.634  0.11792   
## ---
## Signif. codes:  0 '***' 0.001 '**' 0.01 '*' 0.05 '.' 0.1 ' ' 1
## 
## Residual standard error: 80.79 on 20 degrees of freedom
## Multiple R-squared:  0.1178, Adjusted R-squared:  0.07365 
## F-statistic:  2.67 on 1 and 20 DF,  p-value: 0.1179
\end{verbatim}

\begin{Shaded}
\begin{Highlighting}[]
\NormalTok{newR.sq }\OtherTok{\textless{}{-}} \FloatTok{0.1178} \CommentTok{\# From summary of new linear model}
\NormalTok{chi.sq }\OtherTok{\textless{}{-}}\NormalTok{ newR.sq }\SpecialCharTok{*}\NormalTok{ n }\CommentTok{\# 22 observations}

\NormalTok{crit.X }\OtherTok{\textless{}{-}} \FunctionTok{qchisq}\NormalTok{(}\FloatTok{0.95}\NormalTok{, }\DecValTok{1}\NormalTok{, }\AttributeTok{lower.tail=}\ConstantTok{TRUE}\NormalTok{)}

\CommentTok{\# Output}
\NormalTok{BP.out }\OtherTok{\textless{}{-}} \FunctionTok{cbind}\NormalTok{(chi.sq, crit.X)}
\FunctionTok{colnames}\NormalTok{(BP.out) }\OtherTok{\textless{}{-}} \FunctionTok{c}\NormalTok{(}\StringTok{"Chi{-}Sq*"}\NormalTok{, }\StringTok{"Crit. Value"}\NormalTok{)}
\FunctionTok{rownames}\NormalTok{(BP.out) }\OtherTok{\textless{}{-}} \FunctionTok{c}\NormalTok{(}\StringTok{"BP Test"}\NormalTok{)}
\FunctionTok{round}\NormalTok{(BP.out, }\DecValTok{4}\NormalTok{)}
\end{Highlighting}
\end{Shaded}

\begin{verbatim}
##         Chi-Sq* Crit. Value
## BP Test  2.5916      3.8415
\end{verbatim}

\begin{enumerate}
\def\labelenumi{\roman{enumi})}
\tightlist
\item
  F-test for lack-of-fit
\end{enumerate}

H0: E(Yi) = B0 + B1Xi - there is no lack of fit in the model.

Ha: E(Yi) != B0 + B1Xi - there is lack of fit in the model.

We have the test statistic F* = 7.8407. Our p-value is 0.0007684.

Since 0.0007684 \textless{} 0.05, we reject H0, and thus conclude that
there is lack of fit in the model at the significance level 0.05.

\begin{Shaded}
\begin{Highlighting}[]
\CommentTok{\# Full vs. reduced}
\NormalTok{lm\_exp\_data\_full }\OtherTok{\textless{}{-}} \FunctionTok{lm}\NormalTok{(Y }\SpecialCharTok{\textasciitilde{}} \FunctionTok{factor}\NormalTok{(X))}
\NormalTok{lm\_exp\_data\_reduced }\OtherTok{\textless{}{-}} \FunctionTok{lm}\NormalTok{(Y }\SpecialCharTok{\textasciitilde{}}\NormalTok{ X)}
\FunctionTok{anova}\NormalTok{(lm\_exp\_data\_reduced, lm\_exp\_data\_full)}
\end{Highlighting}
\end{Shaded}

\begin{verbatim}
## Analysis of Variance Table
## 
## Model 1: Y ~ X
## Model 2: Y ~ factor(X)
##   Res.Df     RSS Df Sum of Sq      F    Pr(>F)    
## 1     20 1076.63                                  
## 2     14  246.92  6    829.72 7.8407 0.0007684 ***
## ---
## Signif. codes:  0 '***' 0.001 '**' 0.01 '*' 0.05 '.' 0.1 ' ' 1
\end{verbatim}

\begin{enumerate}
\def\labelenumi{\alph{enumi})}
\setcounter{enumi}{9}
\tightlist
\item
  Obtain ``best'' power transformation method, based on Box-Cox
  transformations
\end{enumerate}

First, we run series of power transformations and find the ``best''
lambda value, which is -0.2222. This lambda can be used to fit a new
linear regression model, which relates (Y\^{}lambda-1) / lambda to X,
where lambda is -0.2222.

In R, the model is fitted by lm((Y\^{}lambda-1) / lambda
\textasciitilde{} X).

\begin{Shaded}
\begin{Highlighting}[]
\CommentTok{\# Runs series of power transformations and plots}
\NormalTok{lm\_exp\_data\_BC }\OtherTok{\textless{}{-}} \FunctionTok{boxcox}\NormalTok{(lm\_exp\_data, }\AttributeTok{plot=}\NormalTok{T)}
\end{Highlighting}
\end{Shaded}

\includegraphics{Homework-2_files/figure-latex/unnamed-chunk-13-1.pdf}

\begin{Shaded}
\begin{Highlighting}[]
\CommentTok{\# Print out "best" lambda (max lambda value)}
\NormalTok{lambda }\OtherTok{\textless{}{-}}\NormalTok{ lm\_exp\_data\_BC}\SpecialCharTok{$}\NormalTok{x[}\FunctionTok{which.max}\NormalTok{(lm\_exp\_data\_BC}\SpecialCharTok{$}\NormalTok{y)]}

\NormalTok{BC.out }\OtherTok{\textless{}{-}} \FunctionTok{cbind}\NormalTok{(lambda)}
\FunctionTok{colnames}\NormalTok{(BC.out) }\OtherTok{\textless{}{-}} \FunctionTok{c}\NormalTok{(}\StringTok{"Best Lambda"}\NormalTok{)}
\FunctionTok{rownames}\NormalTok{(BC.out) }\OtherTok{\textless{}{-}} \FunctionTok{c}\NormalTok{(}\StringTok{"Box{-}Cox"}\NormalTok{)}
\FunctionTok{round}\NormalTok{(BC.out, }\DecValTok{4}\NormalTok{)}
\end{Highlighting}
\end{Shaded}

\begin{verbatim}
##         Best Lambda
## Box-Cox     -0.2222
\end{verbatim}

\begin{enumerate}
\def\labelenumi{\alph{enumi})}
\setcounter{enumi}{10}
\tightlist
\item
  Obtain simultaneous 95\% Confidence Intervals for β0, β1
\end{enumerate}

The 95\% confidence interval for β0 is (178.3546, 190.5168). We are 95\%
confident the true value for β0 (the intercept of the regression line)
falls between 178.3546 and 190.5168.

The 95\% confidence interval for β1 is (-0.7659, -0.6248). We are 95\%
confident the true value for β1 (the mean change) falls between -0.7659
and -0.6248.

\begin{Shaded}
\begin{Highlighting}[]
\CommentTok{\# Calculation by hand}
\NormalTok{SSE }\OtherTok{\textless{}{-}} \FunctionTok{sum}\NormalTok{((Y}\SpecialCharTok{{-}}\NormalTok{yhat)}\SpecialCharTok{\^{}}\DecValTok{2}\NormalTok{)}
\NormalTok{MSE }\OtherTok{\textless{}{-}}\NormalTok{ SSE }\SpecialCharTok{/}\NormalTok{ (n }\SpecialCharTok{{-}} \DecValTok{2}\NormalTok{) }\CommentTok{\# n = 22}
\NormalTok{s.b1 }\OtherTok{\textless{}{-}} \FunctionTok{sqrt}\NormalTok{(MSE) }\SpecialCharTok{/} \FunctionTok{sqrt}\NormalTok{(SSxx)}
\NormalTok{s.b0 }\OtherTok{\textless{}{-}} \FunctionTok{sqrt}\NormalTok{(MSE}\SpecialCharTok{*}\NormalTok{(}\DecValTok{1}\SpecialCharTok{/}\NormalTok{n }\SpecialCharTok{+}\NormalTok{ (xbar)}\SpecialCharTok{\^{}}\DecValTok{2} \SpecialCharTok{/}\NormalTok{ SSxx))}

\NormalTok{t.value }\OtherTok{\textless{}{-}} \FunctionTok{qt}\NormalTok{(}\FloatTok{0.975}\NormalTok{, n}\DecValTok{{-}2}\NormalTok{)}

\CommentTok{\# b0 CI}
\NormalTok{b0.LL }\OtherTok{\textless{}{-}}\NormalTok{ b0 }\SpecialCharTok{{-}}\NormalTok{ t.value}\SpecialCharTok{*}\NormalTok{s.b0}
\NormalTok{b0.UL }\OtherTok{\textless{}{-}}\NormalTok{ b0 }\SpecialCharTok{+}\NormalTok{ t.value}\SpecialCharTok{*}\NormalTok{s.b0}

\NormalTok{CI.b0.out }\OtherTok{\textless{}{-}} \FunctionTok{cbind}\NormalTok{(b0.LL, b0.UL)}
\FunctionTok{colnames}\NormalTok{(CI.b0.out) }\OtherTok{\textless{}{-}} \FunctionTok{c}\NormalTok{(}\StringTok{"Lower Bound"}\NormalTok{, }\StringTok{"Upper Bound"}\NormalTok{)}
\FunctionTok{rownames}\NormalTok{(CI.b0.out) }\OtherTok{\textless{}{-}} \FunctionTok{c}\NormalTok{(}\StringTok{"95\% CI for b0"}\NormalTok{)}
\FunctionTok{round}\NormalTok{(CI.b0.out, }\DecValTok{4}\NormalTok{)}
\end{Highlighting}
\end{Shaded}

\begin{verbatim}
##               Lower Bound Upper Bound
## 95% CI for b0    178.3546    190.5168
\end{verbatim}

\begin{Shaded}
\begin{Highlighting}[]
\CommentTok{\# b1 CI}
\NormalTok{b1.LL }\OtherTok{\textless{}{-}}\NormalTok{ b1 }\SpecialCharTok{{-}}\NormalTok{ t.value}\SpecialCharTok{*}\NormalTok{s.b1}
\NormalTok{b1.UL }\OtherTok{\textless{}{-}}\NormalTok{ b1 }\SpecialCharTok{+}\NormalTok{ t.value}\SpecialCharTok{*}\NormalTok{s.b1}

\NormalTok{CI.b1.out }\OtherTok{\textless{}{-}} \FunctionTok{cbind}\NormalTok{(b1.LL, b1.UL)}
\FunctionTok{colnames}\NormalTok{(CI.b1.out) }\OtherTok{\textless{}{-}} \FunctionTok{c}\NormalTok{(}\StringTok{"Lower Bound"}\NormalTok{, }\StringTok{"Upper Bound"}\NormalTok{)}
\FunctionTok{rownames}\NormalTok{(CI.b1.out) }\OtherTok{\textless{}{-}} \FunctionTok{c}\NormalTok{(}\StringTok{"95\% CI for b1"}\NormalTok{)}
\FunctionTok{round}\NormalTok{(CI.b1.out, }\DecValTok{4}\NormalTok{)}
\end{Highlighting}
\end{Shaded}

\begin{verbatim}
##               Lower Bound Upper Bound
## 95% CI for b1     -0.7659     -0.6248
\end{verbatim}

\begin{enumerate}
\def\labelenumi{\alph{enumi})}
\setcounter{enumi}{11}
\tightlist
\item
  Obtain an approximate 95\% Prediction Interval for the Area of the
  coupling, when a deflection of 115 was observed (See section 4.6)
\end{enumerate}

The 95\% prediction interval for Xhat is (77.1829, 122.5269). We are
95\% confident the true value for Xhat falls between 77.1829 and
122.5269 when Yh = 115.

\begin{Shaded}
\begin{Highlighting}[]
\NormalTok{y.h }\OtherTok{\textless{}{-}} \DecValTok{115}
\NormalTok{xhat.h }\OtherTok{\textless{}{-}}\NormalTok{ (y.h }\SpecialCharTok{{-}}\NormalTok{ b0) }\SpecialCharTok{/}\NormalTok{ b1 }\CommentTok{\# Point estimate}
\NormalTok{s.pred }\OtherTok{\textless{}{-}} \FunctionTok{sqrt}\NormalTok{((MSE }\SpecialCharTok{/}\NormalTok{ b1}\SpecialCharTok{\^{}}\DecValTok{2}\NormalTok{) }\SpecialCharTok{*}\NormalTok{ (}\DecValTok{1} \SpecialCharTok{+} \DecValTok{1}\SpecialCharTok{/}\NormalTok{n }\SpecialCharTok{+}\NormalTok{ ((xhat.h }\SpecialCharTok{{-}}\NormalTok{ xbar)}\SpecialCharTok{\^{}}\DecValTok{2} \SpecialCharTok{/}\NormalTok{ SSxx)))}
\NormalTok{t.value }\OtherTok{\textless{}{-}} \FunctionTok{qt}\NormalTok{(}\FloatTok{0.975}\NormalTok{, n}\DecValTok{{-}2}\NormalTok{)}

\CommentTok{\# 95\% CI for Xhat.h, given Yh = 115}
\NormalTok{PI.Xhat.LL }\OtherTok{\textless{}{-}}\NormalTok{ xhat.h }\SpecialCharTok{{-}}\NormalTok{ (t.value}\SpecialCharTok{*}\NormalTok{s.pred)}
\NormalTok{PI.Xhat.UL }\OtherTok{\textless{}{-}}\NormalTok{ xhat.h }\SpecialCharTok{+}\NormalTok{ (t.value}\SpecialCharTok{*}\NormalTok{s.pred)}

\NormalTok{PI.Xhat.out }\OtherTok{\textless{}{-}} \FunctionTok{cbind}\NormalTok{(PI.Xhat.LL, PI.Xhat.UL)}
\FunctionTok{colnames}\NormalTok{(PI.Xhat.out) }\OtherTok{\textless{}{-}} \FunctionTok{c}\NormalTok{(}\StringTok{"Lower Bound"}\NormalTok{, }\StringTok{"Upper Bound"}\NormalTok{)}
\FunctionTok{rownames}\NormalTok{(PI.Xhat.out) }\OtherTok{\textless{}{-}} \FunctionTok{c}\NormalTok{(}\StringTok{"95\% PI for Xhat.h, given Yh = 115"}\NormalTok{)}
\FunctionTok{round}\NormalTok{(PI.Xhat.out, }\DecValTok{4}\NormalTok{)}
\end{Highlighting}
\end{Shaded}

\begin{verbatim}
##                                   Lower Bound Upper Bound
## 95% PI for Xhat.h, given Yh = 115     77.1829    122.5269
\end{verbatim}

\begin{enumerate}
\def\labelenumi{\alph{enumi})}
\setcounter{enumi}{12}
\tightlist
\item
  Obtain X'X, X'Y, (X'X)-1, beta-hat, MSE, and s2\{beta-hat\}
\end{enumerate}

\begin{Shaded}
\begin{Highlighting}[]
\NormalTok{Xmat }\OtherTok{\textless{}{-}} \FunctionTok{matrix}\NormalTok{(}\FunctionTok{c}\NormalTok{(}\FunctionTok{rep}\NormalTok{(}\DecValTok{1}\NormalTok{,n),X), }\AttributeTok{ncol=}\DecValTok{2}\NormalTok{) }\CommentTok{\# X is nx2 matrix [1\_n, X]}
\NormalTok{Ymat }\OtherTok{\textless{}{-}} \FunctionTok{matrix}\NormalTok{(Y, }\AttributeTok{ncol=}\DecValTok{1}\NormalTok{) }\CommentTok{\# Y is nx1 matrix}

\CommentTok{\# Matrix calculations}
\NormalTok{XprimeX }\OtherTok{\textless{}{-}} \FunctionTok{t}\NormalTok{(Xmat) }\SpecialCharTok{\%*\%}\NormalTok{ Xmat }\CommentTok{\# X\textquotesingle{}X }
\NormalTok{XprimeY }\OtherTok{\textless{}{-}} \FunctionTok{t}\NormalTok{(Xmat) }\SpecialCharTok{\%*\%}\NormalTok{ Ymat }\CommentTok{\# X\textquotesingle{}Y}
\NormalTok{invXprimeX }\OtherTok{\textless{}{-}} \FunctionTok{solve}\NormalTok{(XprimeX) }\CommentTok{\# (X\textquotesingle{}X)\^{}({-}1)}

\CommentTok{\# Identity matrix and H (projection) matrix}
\NormalTok{H }\OtherTok{\textless{}{-}}\NormalTok{ Xmat }\SpecialCharTok{\%*\%}\NormalTok{ invXprimeX }\SpecialCharTok{\%*\%} \FunctionTok{t}\NormalTok{(Xmat)}
\NormalTok{ID.mat }\OtherTok{\textless{}{-}} \FunctionTok{diag}\NormalTok{(n)}
\NormalTok{SSE.mat }\OtherTok{\textless{}{-}} \FunctionTok{t}\NormalTok{(Ymat) }\SpecialCharTok{\%*\%}\NormalTok{ (ID.mat }\SpecialCharTok{{-}}\NormalTok{ H) }\SpecialCharTok{\%*\%}\NormalTok{ Y}

\CommentTok{\# Beta{-}hat, MSE, s\^{}2\{beta{-}hat\}}
\NormalTok{bhat }\OtherTok{\textless{}{-}}\NormalTok{ invXprimeX }\SpecialCharTok{\%*\%}\NormalTok{ XprimeY }\CommentTok{\# Beta{-}hat}
\NormalTok{MSE.mat }\OtherTok{\textless{}{-}}\NormalTok{ SSE.mat }\SpecialCharTok{/}\NormalTok{ (n}\DecValTok{{-}2}\NormalTok{)}
\NormalTok{s2.bhat }\OtherTok{\textless{}{-}}\NormalTok{ MSE.mat[}\DecValTok{1}\NormalTok{,}\DecValTok{1}\NormalTok{] }\SpecialCharTok{*}\NormalTok{ invXprimeX }\CommentTok{\# s2\{beta{-}hat\}}

\CommentTok{\# Output matrices}
\FunctionTok{cat}\NormalTok{(}\StringTok{"}\SpecialCharTok{\textbackslash{}n}\StringTok{"}\NormalTok{, }\StringTok{"X\textquotesingle{}X: "}\NormalTok{, }\StringTok{"}\SpecialCharTok{\textbackslash{}n}\StringTok{"}\NormalTok{) }\CommentTok{\# X\textquotesingle{}X}
\end{Highlighting}
\end{Shaded}

\begin{verbatim}
## 
##  X'X:
\end{verbatim}

\begin{Shaded}
\begin{Highlighting}[]
\NormalTok{XprimeX}
\end{Highlighting}
\end{Shaded}

\begin{verbatim}
##      [,1]   [,2]
## [1,]   22   1600
## [2,] 1600 163412
\end{verbatim}

\begin{Shaded}
\begin{Highlighting}[]
\FunctionTok{cat}\NormalTok{(}\StringTok{"}\SpecialCharTok{\textbackslash{}n}\StringTok{"}\NormalTok{, }\StringTok{"X\textquotesingle{}Y: "}\NormalTok{, }\StringTok{"}\SpecialCharTok{\textbackslash{}n}\StringTok{"}\NormalTok{) }\CommentTok{\# X\textquotesingle{}Y}
\end{Highlighting}
\end{Shaded}

\begin{verbatim}
## 
##  X'Y:
\end{verbatim}

\begin{Shaded}
\begin{Highlighting}[]
\NormalTok{XprimeY}
\end{Highlighting}
\end{Shaded}

\begin{verbatim}
##        [,1]
## [1,]   2945
## [2,] 181466
\end{verbatim}

\begin{Shaded}
\begin{Highlighting}[]
\FunctionTok{cat}\NormalTok{(}\StringTok{"}\SpecialCharTok{\textbackslash{}n}\StringTok{"}\NormalTok{, }\StringTok{"(X\textquotesingle{}X)\^{}{-}1: "}\NormalTok{, }\StringTok{"}\SpecialCharTok{\textbackslash{}n}\StringTok{"}\NormalTok{) }\CommentTok{\# (X\textquotesingle{}X)\^{}{-}1}
\end{Highlighting}
\end{Shaded}

\begin{verbatim}
## 
##  (X'X)^-1:
\end{verbatim}

\begin{Shaded}
\begin{Highlighting}[]
\NormalTok{invXprimeX}
\end{Highlighting}
\end{Shaded}

\begin{verbatim}
##              [,1]          [,2]
## [1,]  0.157876228 -1.545798e-03
## [2,] -0.001545798  2.125472e-05
\end{verbatim}

\begin{Shaded}
\begin{Highlighting}[]
\FunctionTok{cat}\NormalTok{(}\StringTok{"}\SpecialCharTok{\textbackslash{}n}\StringTok{"}\NormalTok{, }\StringTok{"beta{-}hat: "}\NormalTok{, }\StringTok{"}\SpecialCharTok{\textbackslash{}n}\StringTok{"}\NormalTok{) }\CommentTok{\# Beta{-}hat}
\end{Highlighting}
\end{Shaded}

\begin{verbatim}
## 
##  beta-hat:
\end{verbatim}

\begin{Shaded}
\begin{Highlighting}[]
\NormalTok{bhat}
\end{Highlighting}
\end{Shaded}

\begin{verbatim}
##             [,1]
## [1,] 184.4356871
## [2,]  -0.6953657
\end{verbatim}

\begin{Shaded}
\begin{Highlighting}[]
\FunctionTok{cat}\NormalTok{(}\StringTok{"}\SpecialCharTok{\textbackslash{}n}\StringTok{"}\NormalTok{, }\StringTok{"MSE: "}\NormalTok{, }\StringTok{"}\SpecialCharTok{\textbackslash{}n}\StringTok{"}\NormalTok{) }\CommentTok{\# MSE}
\end{Highlighting}
\end{Shaded}

\begin{verbatim}
## 
##  MSE:
\end{verbatim}

\begin{Shaded}
\begin{Highlighting}[]
\NormalTok{MSE.mat}
\end{Highlighting}
\end{Shaded}

\begin{verbatim}
##          [,1]
## [1,] 53.83166
\end{verbatim}

\begin{Shaded}
\begin{Highlighting}[]
\FunctionTok{cat}\NormalTok{(}\StringTok{"}\SpecialCharTok{\textbackslash{}n}\StringTok{"}\NormalTok{, }\StringTok{"s\^{}2\{beta{-}hat\}: "}\NormalTok{, }\StringTok{"}\SpecialCharTok{\textbackslash{}n}\StringTok{"}\NormalTok{) }\CommentTok{\# s\^{}2\{beta{-}hat\}}
\end{Highlighting}
\end{Shaded}

\begin{verbatim}
## 
##  s^2{beta-hat}:
\end{verbatim}

\begin{Shaded}
\begin{Highlighting}[]
\NormalTok{s2.bhat}
\end{Highlighting}
\end{Shaded}

\begin{verbatim}
##             [,1]         [,2]
## [1,]  8.49873934 -0.083212879
## [2,] -0.08321288  0.001144177
\end{verbatim}

\end{document}
